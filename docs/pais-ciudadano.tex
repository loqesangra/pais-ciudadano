% Options for packages loaded elsewhere
\PassOptionsToPackage{unicode}{hyperref}
\PassOptionsToPackage{hyphens}{url}
%
\documentclass[
]{book}
\usepackage{amsmath,amssymb}
\usepackage{lmodern}
\usepackage{ifxetex,ifluatex}
\ifnum 0\ifxetex 1\fi\ifluatex 1\fi=0 % if pdftex
  \usepackage[T1]{fontenc}
  \usepackage[utf8]{inputenc}
  \usepackage{textcomp} % provide euro and other symbols
\else % if luatex or xetex
  \usepackage{unicode-math}
  \defaultfontfeatures{Scale=MatchLowercase}
  \defaultfontfeatures[\rmfamily]{Ligatures=TeX,Scale=1}
\fi
% Use upquote if available, for straight quotes in verbatim environments
\IfFileExists{upquote.sty}{\usepackage{upquote}}{}
\IfFileExists{microtype.sty}{% use microtype if available
  \usepackage[]{microtype}
  \UseMicrotypeSet[protrusion]{basicmath} % disable protrusion for tt fonts
}{}
\makeatletter
\@ifundefined{KOMAClassName}{% if non-KOMA class
  \IfFileExists{parskip.sty}{%
    \usepackage{parskip}
  }{% else
    \setlength{\parindent}{0pt}
    \setlength{\parskip}{6pt plus 2pt minus 1pt}}
}{% if KOMA class
  \KOMAoptions{parskip=half}}
\makeatother
\usepackage{xcolor}
\IfFileExists{xurl.sty}{\usepackage{xurl}}{} % add URL line breaks if available
\IfFileExists{bookmark.sty}{\usepackage{bookmark}}{\usepackage{hyperref}}
\hypersetup{
  pdftitle={País Ciudadano},
  pdfauthor={loqesangra},
  hidelinks,
  pdfcreator={LaTeX via pandoc}}
\urlstyle{same} % disable monospaced font for URLs
\usepackage{longtable,booktabs,array}
\usepackage{calc} % for calculating minipage widths
% Correct order of tables after \paragraph or \subparagraph
\usepackage{etoolbox}
\makeatletter
\patchcmd\longtable{\par}{\if@noskipsec\mbox{}\fi\par}{}{}
\makeatother
% Allow footnotes in longtable head/foot
\IfFileExists{footnotehyper.sty}{\usepackage{footnotehyper}}{\usepackage{footnote}}
\makesavenoteenv{longtable}
\usepackage{graphicx}
\makeatletter
\def\maxwidth{\ifdim\Gin@nat@width>\linewidth\linewidth\else\Gin@nat@width\fi}
\def\maxheight{\ifdim\Gin@nat@height>\textheight\textheight\else\Gin@nat@height\fi}
\makeatother
% Scale images if necessary, so that they will not overflow the page
% margins by default, and it is still possible to overwrite the defaults
% using explicit options in \includegraphics[width, height, ...]{}
\setkeys{Gin}{width=\maxwidth,height=\maxheight,keepaspectratio}
% Set default figure placement to htbp
\makeatletter
\def\fps@figure{htbp}
\makeatother
\setlength{\emergencystretch}{3em} % prevent overfull lines
\providecommand{\tightlist}{%
  \setlength{\itemsep}{0pt}\setlength{\parskip}{0pt}}
\setcounter{secnumdepth}{5}
\usepackage{booktabs}
\usepackage{amsthm}
\makeatletter
\def\thm@space@setup{%
  \thm@preskip=8pt plus 2pt minus 4pt
  \thm@postskip=\thm@preskip
}
\makeatother
\ifluatex
  \usepackage{selnolig}  % disable illegal ligatures
\fi
\usepackage[]{natbib}
\bibliographystyle{apalike}

\title{País Ciudadano}
\author{loqesangra}
\date{2021-05-11}

\begin{document}
\maketitle

{
\setcounter{tocdepth}{1}
\tableofcontents
}
\hypertarget{declaraciuxf3n-de-principios}{%
\chapter{Declaración de principios}\label{declaraciuxf3n-de-principios}}

País Ciudadano es un libro:

\begin{itemize}
\tightlist
\item
  Open Source
\item
  Online
\item
  Ciudadano
\item
  Participativo
\end{itemize}

El objetivo es tener una referencia online multidisciplinaria de un modelo de país que todavía no se ha pensado. Si los fundadores de las patrias modernas hubieran tenido las herramientas que tenemos hoy, ¿habrían fundado los países tal y como los conocemos?

Si tenés poco tiempo te recomendamos por lo menos leer la \protect\hyperlink{intro}{Introducción y resumen general}. Te lleva 5 minutos como mucho.

El código fuente del sitio se encuentra en \url{https://github.com/loqesangra/pais-ciudadano}.

\hypertarget{lineamientos}{%
\section{Lineamientos}\label{lineamientos}}

\begin{itemize}
\tightlist
\item
  El libro no es de nadie y es de tod@s a la vez
\item
  La colaboración esta abierta a cualquiera que quiera aportar
\item
  Será a la vez un espacio político sin chicanas políticas
\item
  Será una fuente de conocimiento y referencia para el presente y el futuro
\item
  Será una fuente de educación al explicar y mostrar procesos y metodologías de trabajo que pueden aplicarse en otras áreas
\item
  Será un lugar limpio de la toxicidad reinante en redes sociales y otras plataformas
\end{itemize}

\hypertarget{hashtag}{%
\section{Hashtag}\label{hashtag}}

Desde ya que además de los objetivos planteados este proyecto tiene la esperanza de impactar la realidad en algún momento. Se alienta el uso del hashtag \textbf{\#paisciudadano} en las redes para referirse y comunicar sobre el proyecto. Con el solo hecho de compartir, ya estás colaborando. Inundemos las redes sociales para desperdigar este mensaje.

Se espera que los mensajes generados en redes sociales con este hashtag se abstengan a los siguientes ``Valores y reglas de estilo y conducta'':

\hypertarget{valores-y-conducta}{%
\section{Valores y conducta}\label{valores-y-conducta}}

Inspiradas por la \href{https://www.reddit.com/wiki/es/reddiquette}{Reddiqueta} o \href{https://reddit.zendesk.com/hc/en-us/articles/205926439-Reddiquette}{Reddiquette}.

\begin{itemize}
\tightlist
\item
  Recordá que quien lee lo que escribís es un ser humano
\item
  Comportate en internet como te comportarías en la vida real
\item
  Leé las reglas antes de colaborar
\item
  Leé la documentación proveída, para algo está
\item
  La moderación se hará con respecto a la calidad del contenido, no a las opiniones
\item
  Las opiniones deberán exponerse desde un lugar individual, apartidario y fáctico en la medida de lo posible
\item
  Usa la gramática y la ortografía correctamente. Se permiten deformaciones del lenguaje siempre y cuando estén en función de la comunicación.
\item
  Es obligatorio citar fuentes si estás utilizando algún recurso externo
\item
  Las críticas deben ser constructivas
\item
  Tratemos de conservar la buena onda. Para mala onda ya es suficiente con la realidad.
\end{itemize}

Ver ``Código de Conducta'' para una expansión respecto a estos conceptos.

\hypertarget{cuxf3mo-colaborar}{%
\section{Cómo colaborar}\label{cuxf3mo-colaborar}}

Una de las iniciativas será escribir una guía para que cualquiera sin conocimientos en desarrollo web, programación o cualquiera de las tecnologías utilizadas por este libro pueda hacerlo. Por el momento la única referencia con respecto a como colaborar será el proyecto original: \url{https://github.com/rstudio/bookdown}.

Si pasás por acá y querés ayudarme por ahora por favor escribime a \href{mailto:loqesangra@gmail.com}{\nolinkurl{loqesangra@gmail.com}}. Algo para hacer te voy a pasar, seguro.

El libro y proyecto es abierto y público en todo sentido. No se pretende tener ninguna línea editorial ni bajada de línea de ningún tipo. Así mismo su propia estructura irá modificándose y siendo maleable con el tiempo.

\hypertarget{lo-que-muxe1s-necesita-este-proyecto-en-este-momento}{%
\section{Lo que más necesita este proyecto en este momento}\label{lo-que-muxe1s-necesita-este-proyecto-en-este-momento}}

\begin{itemize}
\tightlist
\item
  Colaboradores de cualquier tipo
\item
  Ayuda editorial para definir mejor los contenidos y lineamientos
\item
  Algún tipo de gerencia de proyecto para definir prioridades y roadmap
\item
  Ayuda técnica para lidiar con las herramientas tecnológicas
\end{itemize}

No necesitás ningún tipo de experiencia previa ni credenciales. Solo pensar.

\hypertarget{posibles-capuxedtulos-y-temuxe1ticas-a-ser-escritas}{%
\section{Posibles capítulos y temáticas a ser escritas}\label{posibles-capuxedtulos-y-temuxe1ticas-a-ser-escritas}}

Sin ningún orden particular, ideas aleatorias de desarrollo para el proyecto, desde ya y sin temor a repetir, el proyecto necesita tu ayuda para llevar a cabo todo esto:

Posibles textos o capítulos:

\begin{itemize}
\tightlist
\item
  Un Estado

  \begin{itemize}
  \tightlist
  \item
    Contenidos de qué debería tener un estado ideal, de qué es lo que se está haciendo bien y qué mal
  \end{itemize}
\item
  Desarmar la radicalización de los discursos existentes explicar que cualquier partido político que se endilgue la potestad de ser ``la solución'' a determinada realidad está equivocado, ya que no hay soluciones absolutas ni contextos fijos
\item
  Fomentar el respeto y la inclusión por los demás
\item
  Individualismo y colectivismo (por qué necesitamos de los dos, del desarrollo individual y del desarrollo colectivo)
\item
  Lo importante de tener foco para progresar en algo, de tener metas a largo y corto plazo y cómo esto mismo aplica a los países
\item
  Corrupción y transparencia

  \begin{itemize}
  \tightlist
  \item
    ¿El gran problema? Definitivamente no el único, pero es uno de los más grandes? Es una urgencia que se tiene que desarmar ya. No podemos permitir más corrupción. Cómo se evita? Con herramientas transparentes. Hoy el mundo online nos permite tener estas herramientas. Este proyecto es un ejemplo de transparencia.
  \end{itemize}
\item
  Permeabilidad

  \begin{itemize}
  \tightlist
  \item
    cómo las instituciones, constituciones y leyes clásicas ya son algo anacrónico y podemos modernizarlas
  \end{itemize}
\item
  Ideologías

  \begin{itemize}
  \tightlist
  \item
    fueron funcionales en un momento de la historia, centrémonos en la libertad y el respeto por el otro (después veremos cómo esto aplica o no aplica a la economía)
  \end{itemize}
\item
  Redes sociales y su uso

  \begin{itemize}
  \tightlist
  \item
    cómo estamos perdiendo tiempo todos discutiendo por TW, Reddit, IG, etc, etc, en vez de generar contenido de valor que podría estar cambiando el curso de la historia
  \end{itemize}
\item
  Tecnología y su uso y abuso
\item
  Revisar y reflexionar sobre palabras y términos que se usan con una soltura e ignorancia absolutas, generando confusión en la ciudadanía
\item
  Análisis de los modelos de Estado actuales y en qué podrían ser más eficientes (y cómo, definir procesos, etc)
\item
  Recursos de educación financiera adaptados a las realidades y posibilidades de cada un@
\end{itemize}

\hypertarget{intro}{%
\chapter{Introducción y resumen general}\label{intro}}

Se hace largo leer hoy en día, lo sabemos. Pero por favor, tomate 5 minutos y leé todo esto.

Entre tantas maravillas que han podido crear los seres humanos a lo largo de su historia una de ellas es, sin lugar a dudas, la escritura. Los libros, las palabras, los discursos, han movido el curso de La Historia.

La política actual se ha apuntalado a base de discursos y palabras. Cuando debería ser al revés, res non verba (hechos, no palabras).

\textbf{Necesitamos un discurso centralizado, político pero apartidario, donde diversas fuerzas ciudadanas, individuales y colectivas, puedan encontrar puntos de encuentro.}

Este libro pretende ser eso (es abierto, público y colaborativo - ver \protect\hyperlink{declaraciuxf3n-de-principios}{Declaración de Principios}).

Este proyecto es encarado con la solemnidad histórica que ha tenido cualquier libro que se precie de serlo, y a la vez, eleva aún más la vara: en estos tiempos donde evolucionamos de manera exponencial, un libro impreso queda anticuado en el preciso momento en que la tinta empapa el papel. Por eso este libro es un organismo vivo. Alguna versión del mismo podrá ser impresa pero en su base \textbf{País Ciudadano es un libro Vivo, Libre y Participativo, en constante transformación.}

Esta idea de un ``libro vivo'' no es nueva y ya se ha implementado exitosamente, algunos ejemplos:

\begin{itemize}
\tightlist
\item
  \url{https://hybridpedagogy.org/\#publishing}
\item
  \url{https://geocompr.github.io/user_19/presentation/\#1}
\item
  \url{https://geocompr.robinlovelace.net/}
\end{itemize}

\hypertarget{quuxe9-anda-pasando-status-quo}{%
\section{Qué anda pasando (Status quo)}\label{quuxe9-anda-pasando-status-quo}}

\begin{quote}
``Status quo:'' expresión latina con que se hace referencia al estado o situación de ciertas cosas, como la economía, las relaciones sociales o la cultura, en un lugar y momentos determinados.
\end{quote}

El ``lugar y momento determinado'' de estas palabras es Argentina 2021 (en la medida en que se sumen colaboradores podrá este análisis expandir sus fronteras).

Estamos convencidos de que hay una Crisis Institucional y de Representatividad. Trataremos de explayarnos sobre esta idea pero por ahora la base es la siguiente: todos los países tienen problemas diferentes. Pero hay un problema que es común a todos y es precisamente esta Crisis.

La teoría sobre la que se va a basar este libro es: \textbf{la causa de gran parte de los problemas reinantes en las sociedades actuales se pueden extrapolar, en la mayoría de los casos, a una crisis institucional y de representatividad}. Esta idea no es nueva y pueden citarse múltiples referencias académicas de la misma. (ejemplo: \url{https://revista-estudios.revistas.deusto.es/article/view/239/376})

\hypertarget{crisis-institucional-y-de-representatividad}{%
\section{Crisis Institucional y de Representatividad}\label{crisis-institucional-y-de-representatividad}}

¿A qué nos referimos con esto? A una idea muy simple. Tan simple que repasando esta definición de Democracia se puede ver:

\begin{quote}
La democracia (del latín tardío democratĭa, y este del griego δημοκρατία dēmokratía) es una manera de organización social que atribuye la titularidad del poder al conjunto de la ciudadanía. En sentido estricto, la democracia es una forma de organización del Estado en la cual las decisiones colectivas son adoptadas por el pueblo mediante mecanismos de participación directa o indirecta que confieren legitimidad a sus representantes. En sentido amplio, democracia es una forma de convivencia social en la que los miembros son libres e iguales y las relaciones sociales se establecen conforme a mecanismos contractuales.
\end{quote}

La Crisis Está porque La Democracia No Está. \textbf{La Democracia está rota en su implementación}. Hoy la ``Democracia'' no le otorga la titularidad del poder al conjunto de la ciudadanía. En repetidos países de todo el mundo los mecanismos de participación directa o indirecta que deberían conferir legitimidad a sus represantes, no están cumpliendo ese objetivo.

Hoy en día ``los representantes'' del pueblo no responden a los intereses del pueblo; responden a los intereses de las grandes concentraciones de poder (fuerzas económicas, corporativas, de comunicación, sociales).

¿Y por qué pasa esto? Porque \textbf{el Modelo de Instituciones que tenemos quedó anticuado y tiene muchos problemas y huecos en su ejecución} que permiten manejos espurios, malversación de fondos públicos, manipulación del discurso público, corrupción de todos los tipos, etc.

\hypertarget{la-soluciuxf3n}{%
\section{La solución}\label{la-soluciuxf3n}}

Habiendo coincidido en que la Crisis Institucional y de Representatividad es el gran problema de este siglo, es fácil inferir que la solución entonces pasa por una reformulación absoluta de Nuestras Instituciones, la forma en que funcionan, la forma de gobierno incluso: una reformulación de todo el status quo.

Tenemos que entender que la forma en que gobernamos y nuestro sistema de gobierno son ``nuestros'' (de la ciudadanía) de manera literal: la ciudadanía define la forma en que se gobierna, la ciudadanía tiene en sus manos arreglar el sistema de gobierno que está roto.

Los ciudadanos podemos y debemos impactar en nuestra realidad. Ya estamos todos cansados. Ya estamos todos en las calles. Falta esta vuelta de rosca. Replantear nuestras instituciones.

La solución a nuestros problemas no pasa por discutir cada vez más y más violentamente, la solución a nuestros problemas no pasa por ver quién tiene razón, la solución a nuestros problemas no pasa por los extremos, la solución a nuestros problemas no pasa por individuos o ideologías.

La solución a nuestros problemas pasa por intercambiar ideas pacíficamente, por sacarse las camisetas políticas y discutir planes de acción, por entender que no hace falta ver quién tiene razón si podemos comprobarlo con estadísticas y datos, por entender que acercándonos y no alejándonos vamos a lograr consenso, por entender que tenemos que hacerlo todos juntos, unidos como sociedad en la medida de lo posible, y entendiendo que, justamente, como la definición de arriba dice:

\textbf{Podemos construir Una Democracia (o Forma de Gobierno) donde realmente se genere una convivencia social y pacífica en la que los miembros sean libres e iguales (iguales ante La Forma, entendiendo que a la vez Somos Únicos y Diferentes) y las relaciones sociales se establezcan conforme a mecanismos contractuales}.

\hypertarget{cuxf3mo-llevar-a-la-pruxe1ctica-esa-soluciuxf3n}{%
\section{Cómo llevar a la práctica esa solución}\label{cuxf3mo-llevar-a-la-pruxe1ctica-esa-soluciuxf3n}}

Hace no tantos años un proyecto de esta magnitud requeriría años de planeamiento, muchísimo dinero atrás, etc. El encuadre clásico sería formar un partido político y hacerlo andar.

Hoy en día tenemos la tecnología necesaria para:

\begin{itemize}
\tightlist
\item
  juntar una gran masa crítica detrás de una iniciativa
\item
  plantear las modificaciones institucionales que sean necesarias
\item
  auditar al gobierno y todos sus miembros
\item
  auditar al Estado y su funcionamiento
\item
  acercar a la realidad esa idea de ``gobierno de la multitud'' como decía Platón o ``gobierno de los más'' como decía Aristóteles
\end{itemize}

Cuando caés en la cuenta de que el sistema de gobierno que utilizamos hoy nació en el siglo XVIII con las revoluciones francesa y estadounidense te cae la ficha.

¿Somos concientes de esto? Nuestro sistema de gobierno fue pensado en el año 1780 aproximadamente.

Hace 240 años.

\textbf{Es hora de repensar cómo gobernamos. Es hora de que una vez más la ciudadanía se haga cargo de su destino}.

Esta vez, a diferencia de muchos otros momentos de la historia, podemos hacerlo pacíficamente apalancando las herramientas tecnológicas que ya tenemos en nuestro poder.

\textbf{De eso se trata \#paísciudadano}. Esta iniciativa que por el momento tiene forma de libro.

Podría ser mucho más.

\hypertarget{ideologuxeda}{%
\chapter{Ideología}\label{ideologuxeda}}

El término ``ideología'' fue formulado por el marqués de Tracy (un francés, filósofo de la Ilustración) en 1801. Recordemos que esto es época de la Revolución Francesa (todo tiene que ver con todo, ¿no?).

Detengámonos en esto un momento y concienticemos: el término ``ideología'' fue inventado, de la nada por una persona. Antes de 1801 nadie había formulado ni el término, ni la idea.

Hagamos fast-forward hasta 1845 (casi 50 años después): Karl Marx y Friedrich Engels, resignifican el término en su obra ``La ideología alemana''). Para ellos ``ideología'' es (según Wikipedia) ``el conjunto de principios que explican el mundo en cada sociedad en función de sus modos de producción, relacionando los conocimientos prácticos necesarios para la vida con el sistema de relaciones sociales; la relación con la realidad es tan importante mantener esas relaciones sociales, y en los sistemas sociales en los que se da alguna clase de explotación, evitar que los oprimidos perciban su estado de opresión.''

Wow. ¿Soy yo o es una definición de una complejidad extraordinaria? A mí lo que me parece más interesante es que es una definición definitivamente ``filosófica'', más que ``política''. Es decir, es filosófica y política a la vez, pero no en cantidades iguales.

A lo que quiero ir con todo esto es que, así como estos pensadores definieron esas ideologías, un pensador hoy podría definir una ideología nueva. De hecho un montón de gente no se siente alineado con ninguna ideología en particular, sino simplemente un conjunto de ideas y valores generales que pueden derivarse de la \href{https://es.wikipedia.org/wiki/Regla_de_oro_(\%C3\%A9tica)}{Regla de oro}.

\hypertarget{asuxed-como-las-ideologuxedas-existen-las-ideologuxedas-no-existen}{%
\section{Así como las ideologías existen, las ideologías no existen}\label{asuxed-como-las-ideologuxedas-existen-las-ideologuxedas-no-existen}}

Este libro no se apega a ninguna ideología porque no le encuentra sentido ni utilidad en el mundo actual. Las ideologías per-se o puras ya no existen. Fueron un conjunto de ideas filosóficas, psicológicas y sociales que en algún momento de la historia pueden haber sido utiles para congregar a las masas en grupos concretos (capitalismo vs comunismo por ejemplo). (Otro problema de las ``ideologías'' es que pueden fácilmente ser utilizadas como medio de manipulación de las masas.)

Aún así hay gente que necesita un encuadre ideológico para sumarse a un proyecto, como decíamos arriba en ese caso el encuadre sería el de la \href{https://es.wikipedia.org/wiki/Regla_de_oro_(\%C3\%A9tica)}{Regla de oro}:

\begin{quote}
Un principio moral general que puede expresarse: «trata a los demás como querrías que te trataran a ti» (en su forma positiva) o «no hagas a los demás lo que no quieras que te hagan a ti» (en su forma negativa, en esta forma también conocida como regla de plata). No consiste en la afirmación de determinadas conductas o en la imposición de valores afirmativos o positivos, como sucede en las doctrinas dogmáticas, sino que preconiza una dinámica de relaciones intersubjetivas basada en el sentido común y en el principio de no agresión.
\end{quote}

\begin{quote}
Su universalidad sugiere que puede estar relacionada con aspectos innatos de la naturaleza humana. Quien la aplique tratará con consideración a todos los seres humanos, y no solo a miembros de su grupo. Se considera a la regla de oro el punto de partida para la reflexión teórica y el proceso histórico que condujo a la formulación de los derechos humanos; aunque identificar ambos conceptos es anacrónico.
\end{quote}

Como dato curioso, esta idea de usar la Regla de Oro para definir cuestiones políticas y sociales no es nueva:

\begin{quote}
John Locke propuso los derechos a «la vida, la libertad y la propiedad». Para Locke, el propio cuerpo es parte de los bienes de un hombre y, por tanto, sobre él se ejerce un derecho a la propiedad que teóricamente garantiza la seguridad de las personas al igual que la de sus posesiones. Posteriormente, este concepto fue recogido por la Ilustración y el pensamiento democrático posterior a la Revolución francesa
\end{quote}

Así, entonces, creemos que, ante todo, el primer punto en común que necesita esta solución ciudadana es el expresado en estas palabras. Algunas citas más que expanden este concepto:

\begin{quote}
George Bernard Shaw (1898) estableció una evidente precaución a la aplicación de la regla de oro en sentido activo o positivo: «no hagas a otros lo que quisieras que te hagan a ti. Sus gustos pueden no ser los mismos»
\end{quote}

\begin{quote}
El filósofo alemán Hans Reiner (1896-1991) distinguía diferentes formulaciones de la regla de oro: la regla de empatía, que parte de nuestros deseos o temores («lo que tú mismo temas, no lo hagas a los demás, lo que deseas para ti, hazlo a los demás»), y la regla de la equidad, que parte de nuestros juicios de valor («lo que reprochas a otros, no lo hagas tú mismo; debes actuar como juzgas que los demás deben hacerlo»)
\end{quote}

\begin{quote}
Thomas Nagel (1970) propuso repensar el altruismo de forma objetiva sobre la base de la ética de la reciprocidad
\end{quote}

\begin{quote}
En los años 1990, Enno Winkler desarrolló un código de ética universal en el que la regla de oro está incluido como un mandamiento para las relaciones interpersonales en ausencia de empatía: «¡Respeta al otro como a ti mismo!».
\end{quote}

\hypertarget{appendix-apuxe9ndice}{%
\appendix}


\hypertarget{codigo-de-conducta}{%
\chapter{Código de conducta}\label{codigo-de-conducta}}

\hypertarget{nuestro-compromiso}{%
\section{Nuestro compromiso}\label{nuestro-compromiso}}

Nosotr@s, miembr@s, contribuidores y líderes nos comprometemos a hacer que la participación en nuestra comunidad sea una experiencia libre de acoso u hostigamiento para tod@s, sin importar la edad, el cuerpo, cualquier tipo de impedimento visible o invisible, etnicidad, sexualidad, identidad u expresión de género, nivel de experiencia, educación, posición socio-económica, nacionalidad, apariencia personal, raza, religión o identidad y orientación sexual.

Nos comprometemos a actuar e interactuar con formas que contribuyan a una comunidad abierta, acogedora, diversa, inclusiva y saludable.

\hypertarget{nuestros-estuxe1ndares}{%
\section{Nuestros estándares}\label{nuestros-estuxe1ndares}}

Ejemplos de un comportamiento que contribuye a un ambiente positivo para nuestra comunidad incluyen:

\begin{itemize}
\tightlist
\item
  Demostrar empatía y amabilidad para con las personas
\item
  Ser respetuoso de otras opiniones, puntos de vistas y experiencias
\item
  Dar y recibir agradecidamente el feedback constructivo
\item
  Aceptar responsabilidades y disculparse para quienes sean afectados por nuestros errors, y aprendiendo de la experiencia
\item
  Enfocarse en lo que es mejor no solo para nosotros como individuos, sino para toda la comunidad
\end{itemize}

Ejemplos de un comportamiento que no será aceptado incluyen:

\begin{itemize}
\tightlist
\item
  El uso de lenguaje o imágenes sexualizadas y cualquier tipo de atención o avance sexual
\item
  Trolling, insultos o comentarios derogativos y cualquier tipo de ataque político o personal
\item
  Acoso público o privado
\item
  Publicación de información privada de terceros sin su permiso explícito
\item
  Cualquier otra conducta que pueda ser razonablemente considerada inapropiada en un entorno profesional
\end{itemize}

\hypertarget{responsabilidades-de-ejecuciuxf3n}{%
\section{Responsabilidades de Ejecución}\label{responsabilidades-de-ejecuciuxf3n}}

L@s líderes de la comunidad son responsables de clarificar y hacer cumplir nuestros estándares de comportamiento aceptado y tomarán acciones correctivas apropiadas y justas en respuesta a cualquier comportamiento que consideren inapropiado, amenzante, ofensivo o dañino.

L@s líderes de la comunidad tienen el derecho y la responsabilidad de remover, editar o rechazar comentarios, commits, código, ediciones de wiki, issues/tickets y cualquier otra contribución que no esté alineada a este Código de Conducta y comunicarán las razones de su acción moderadora de ser necesario.

\hypertarget{alcance}{%
\section{Alcance}\label{alcance}}

Este Código de Conducta aplica a todos los espacios comunitarios y también aplica a un individuo que esté representando oficialmente a la comunidad en espacios públicos. Ejemplos de ``representación'' incluyen utilizar una dirección de e-mail oficial, postear via una cuenta oficial de redes sociales o actuando como un representante en un evento online u offline.

\hypertarget{ejecuciuxf3n}{%
\section{Ejecución}\label{ejecuciuxf3n}}

Las instancias de comportamiento inaceptable, abusivo o acosador pueden ser reportadas a l@s líderes de la comunidad responsables de ejecutar el Código de Conducta. Todas las denuncias serán revisadas e investigadas prontamente y justamente.

Tod@s l@s líderes de la comunidad están obligad@s a respetar la privacidad y seguridad de quien reporte cualquier incidente.

\hypertarget{medidas-de-ejecuciuxf3n}{%
\section{Medidas de Ejecución}\label{medidas-de-ejecuciuxf3n}}

L@s líderes de la comunidad seguirán estas ``Pautas de Impacto en la Comunidad'' para determinar las consecuencias de cualquier acción que consideren viole este Código de Conducta:

\hypertarget{correcciuxf3n}{%
\subsection{CORRECCIÓN}\label{correcciuxf3n}}

Impacto en la Comunidad: Uso de lenguaje inapropiado o cualquier otro comportamiento considerado no profesional o no bienvenido en la comunidad.

Consecuencia: Una advertencia escrita por pivado de parte de l@s líderes de la comunidad, proveyendo claridad respecto a la naturaleza de la violación y una explicación de por qué el comportamiento fue inapropiado. Una disculpa pública puede ser solicitada.

\hypertarget{advertencia}{%
\subsection{ADVERTENCIA}\label{advertencia}}

Impacto en la Comunidad: Una violación a través de un incidente único o una serie de acciones.

Consecuencia: Una advertencia con consecuencias por comportamiento continuo. No se permite la interacción con las personas involucradas, incluyendo interacciones no solicitadas para con quienes hacen cumplir el Código de Conducta, por un período especificado de tiempo. Esto incluye evitar interacciones en espacios comunitarios así como también canales externos como redes sociales. La violación de estos términos puede llevar a un ban temporal o permanente.

\hypertarget{ban-temporal}{%
\subsection{BAN TEMPORAL}\label{ban-temporal}}

Impacto en la Comunidad: Una violación seria de los estándares de la comunidad, incluyendo un comportamiento inapropiado sostenido en el tiempo.

Consecuencia: Un ban temporal de cualquier tipo de interacción o comunicación pública para con la comunidad por un período específico de tiempo. No se permite ninguna interacción pública o privada con las personas involucradas, incluyendo interacciones no solicitadas para con quienes hacen cumplir el Código de Conducta. Violar estos términos puede terminar en un ban permanente.

\hypertarget{ban-permanente}{%
\subsubsection{BAN PERMANENTE}\label{ban-permanente}}

Impacto en la Comunidad: Demostración de patrones de violación de los estándares de la comunidad, incluyendo un comportamiento inapropiado sostenido en el tiempo, acoso de un individuo o agresión o menosprecio por cualquier clase de individuo.

Consecuencia: Un ban permanente de cualquier tipo de interacción con la comunidad.

\hypertarget{atribuciuxf3n}{%
\section{Atribución}\label{atribuciuxf3n}}

Basado en \url{https://pkgs.rstudio.com/bookdown/CODE_OF_CONDUCT.html} que a su vez está basado en el Contributor Covenant, version 2.0, disponible en \url{https://www.contributor-covenant.org/version/2/0/code_of_conduct.html}.

  \bibliography{book.bib,packages.bib}

\end{document}
